\section{Hash Functions}
\label{sec:hash-functions}
A \emph{Hash function} is a function that calculates the hash of input data (i.e., key) of arbitrary size and maps it to a uniform, fixed-size output called hash (also called digest). Hash functions are used in cryptography, checksums, and data partitioning. They are various hash functions developed such as SHA-1 \cite{RFC3174USSecure} and MD5 \cite{rivestRFC1321MD5MessageDigest1992} are developed for these use-cases. Other hash functions like the MurmurHash is a non-cryptographic hash function suitable for general hash-based lookup. MurmurHash2 is used in multiple open source projects. For example, Kafka uses the MurmurHash2 function as its default partitioner\footnote{\url{https://github.com/dpkp/kafka-python/blob/master/kafka/partitioner/default.py}}.


Key-value data use hash functions to partition data on multiple instances. First, an arbitrary hash function (SHA-1, MD5, MurmurHash2) calculates the key's hash. The partitioner should use equidistant range partitioning on the range of hashes. Therefore, the modulo of the particular hash is calculated over the length of the partitions. This calculation yields a partition number for that specific key. This work uses the MurmurHash2 function as its baseline hash function. The recommendation quality of the workers partitioned with the MurmurHash2 function is then assessed with the proposed content-based partitioning method.