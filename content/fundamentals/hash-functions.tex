\section{Hash Functions}
\label{sec:hash-functions}
\begin{itemize}
    \item A hash function is any function that can be used to map data of arbitrary size to fixed-size values. The values returned by a hash function are called hash values, hash codes, digests, or simply hashes.
    \item 
\end{itemize}


\subsection{MurmurHash}
\label{subsec:murmurhash}
\begin{itemize}
    \item MurmurHash is a non-cryptographic hash function suitable for general hash-based lookup. It was created by Austin Appleby in 2008[4] and is currently hosted on GitHub along with its test suite named 'SMHasher'. It also exists in a number of variants, all of which have been released into the public domain. The name comes from two basic operations, multiply (MU) and rotate (R), used in its inner loop.
    \item MurmurHash2 is used in multiple open source projects. Kafka uses the MurmurHash2 function as its default partitioner.
    \item This work uses the kafka-python implementation of the MurmurHash2\footnote{\url{https://github.com/dpkp/kafka-python/blob/master/kafka/partitioner/default.py}} .
\end{itemize}

\todo{do I really need to write all of it down?}
\begin{algorithm}[H]
	\caption{MurmurHash2 algorithm}
	
	\SetAlgoLined
    \SetKwInOut{Input}{input}
    \Input{data (bytes): opaque bytes}
    \KwResult{MurmurHash2 of data}
	
	initialization\;
    length $\leftarrow$ len(data)\;
    seed $\leftarrow$ 0x9747b28c\;
    m $\leftarrow$ 0x5bd1e995\;
    r $\leftarrow$ 24\;
    h $\leftarrow$ seed $\oplus$ length\;
    length4 $\leftarrow$ $\left \lfloor \frac{length}{4} \right \rfloor$ \;
	\While{not at end of this document}{
		read current\;
		\eIf{understand}{
			go to next section\;
			current section becomes this one\;
			}{
			go back to the beginning of current section\;
			}
		}
\end{algorithm}

\subsection{ModuloHash}
\begin{itemize}
    \item One of the simplest and most common methods in practice is the modulo division method.
    \item There is an implementation of this hash function but in this work I prefered the MurmurHash2 function over this approach.
\end{itemize}
