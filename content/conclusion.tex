\chapter{Conclusion}

This work implements a partitioning module that, at its core, uses an embedding model to partition document data based on their topic on different workers. The content-based partitioning idea proposed in this work is prototyped in a recommender system called GraphJet. In the distributed environment, each worker is an instance of GraphJet, implementing the index storage, recommendation engine, and the API endpoint module. The recommendation results of each worker are merged using novel data fusion approaches proposed in this. This work studies and measures the improvements made by scaling the prototyped system horizontally with an evaluation pipeline.


From the beginning of this work, it was evident that the recommendation quality will drop if a worker obtains a fraction of the data. Nevertheless, the results of the evaluation show this drop. The main goal was to generate identical recommendations on a multi-partition environment to the single instance. The results of the experiments denote a higher similarity between the proposed approach compared to the random partitioning technique. The generated baseline with the assumption that all the items of the baseline are relevant made the evaluation strict and affected the overall results. Despite the limitations, these are valuable findings showing that the proposed partitioner beats the random partition method.


This experiment adds to a growing corpus of research showing that the partitioning of the data affects the recommendation generation time. The findings indicate that the indices reduce on each worker, and the bipartite graph shrinks. Therefore, the random-walk algorithm can faster retrieve the adjacency lists of an index.


Latency improvements and possibly even recommendation improvements are the results this work shows and delivers.