\chapter{Conclusion}

This research studies a partitioning module that uses an embedding model to divide text document data on various workers based on their subject. The content-based partitioning idea proposed in this work is prototyped in a recommender system called GraphJet. In the distributed environment, each worker implements the index storage, recommendation engine, and the API endpoint module of GraphJet. Each worker's recommendation results are merged to a single ranked list of items using innovative data fusion techniques described in this thesis. This research studies and measures the various test scenarios using the introduced evaluation pipeline.


From the beginning of this study, it was evident that the recommendation quality will drop if a worker obtains a fraction of the data. Nevertheless, the results of the evaluation show this drop. The main goal was to generate identical recommendations on a multi-partition environment to the single instance. The results of the experiments show that the proposed approach(StarSpace Most Interactions) outperforms the state-of-the-art Murmur2 partitioning. It is noticeable that the performance of StarSpace Most Interactions cannot be considered ideal with an accuracy of 50\%. But it is worth noting that even the Single Partition method offers only 63\% accuracy.


Assuming that all the baseline items are relevant made the evaluation strict and affected the overall results. The metric calculation is directly affected by this assumption. In practice, recommendation data consists of user feedback, determining if the suggested items are of interest or not. Hence, labeling the data was beyond the scope of this research. Despite the limitations, these are valuable findings showing that the proposed partitioner surpasses the random partition method.


The results demonstrate two things. First, the Most Interactions data fusion approach performs well with StarSpace partitioning. StarSpace partitions the data based on its subject, and when a user has a higher degree inside that partition (i.e., more interaction with documents in that partition), it is also likely that this user has more interest in the documents (i.e., tweets) from that partition. 

Second, the Union Results method relies on the hit number for choosing the recommendations. However, the hit number is heavily dependant on the amount of data on each partition. We can see that the Union Results method works better on Murmur2 and performs worse on StarSpace. This is because Murmur2 distributes the data more uniformly across the partitions compared to StarSpace.


 The findings also indicate that the partitioning of the data affects the recommendation generation time. When partitioning the data on multiple workers, the data each worker holds reduce respectively. This leads to a reduction of indices on each worker. The worker maintains a sparse bipartite graph compared to the entire graph. Therefore, the random-walk algorithm can retrieve the adjacency lists of an index more quickly.


This study joins a growing corpus of research showing that latency improvements and possibly even recommendation improvements can be achieved through content-based partitioning.
