\chapter{Conclusion}

This work introduces a partitioning middleware, which learns the partitioning of the data based on the input and output of systems and is entirely independent of the algorithm. The studied method uses an embedding model to divide text document data on various workers based on their subject. The content-based partitioning idea proposed in this work is prototyped in a recommender system called GraphJet. In the distributed environment, each worker implements the index storage, recommendation engine, and the API endpoint module of GraphJet. Each worker's recommendation results are merged to a single ranked list of items using innovative data fusion techniques described in this thesis. This research studies and measures the various test scenarios using the introduced evaluation pipeline.


It is evident that the recommendation quality will drop if a worker obtains a fraction of the data. The main goal was to generate recommendations in a multi-partition environment as identical to the single instance recommendations. The results of the experiments (with 500 sampled users in a two-partition environment) show that the proposed approach (StarSpace Most Interactions) outperforms the baseline partitioning method (Murmur2 Union Results). It is noticeable that the performance of StarSpace Most Interactions cannot be considered ideal with an accuracy of 50\%. On the other hand, the comparison between the single instance and the baseline delivers a 63\% similarity of the results, denoting the not deterministic behavior of the random walk algorithm in producing recommendations.


Assuming that all the items not included in the baseline are irrelevant made the evaluation strict. Some of these items, which are not included in the baseline, might be of higher interest to the user. Moreover, using the Mean Average Precision (MAP) as the primary metric to evaluate the results does not provide enough evidence of whether the results got better or worse. The metric fails to measure if the not relevant data (the items that are not in the baseline) are of interest to the user. Therefore, no definite conclusion can be drawn as to whether the method works or not. 


In practice, recommendation data consists of user feedback, which can be gathered with a user study. The Web UI results already show more promising outcomes for a user, although further studies are needed to prove more evidence. 


Despite the limitations, these are valuable findings showing that the proposed partitioner surpasses the random partition method. The results demonstrate two things. First, the Most Interactions data fusion approach performs well with StarSpace partitioning. StarSpace partitions the data based on its subject, and when a user has a higher degree inside that partition (i.e., more interaction with documents in that partition), it is also likely that this user has more interest in the documents (i.e., tweets) from that partition. 


Second, the Union Results method relies on the hit number for choosing the recommendations. However, the hit number is heavily dependant on the amount of data on each partition. We can see that the Union Results method works better on Murmur2 and performs worse on StarSpace. This is because Murmur2 distributes the data more uniformly across the partitions compared to StarSpace.


The findings also indicate that the partitioning of the data affects the recommendation generation time. When partitioning the data on multiple workers, the data each worker holds reduces respectively. This leads to a reduction of indices on each worker. The worker maintains a sparse bipartite graph compared to the entire graph. Therefore, the random-walk algorithm can retrieve the adjacency lists of an index more quickly.


This study joins a growing corpus of research showing that latency improvements and possibly even recommendation improvements can be achieved through content-based partitioning.
