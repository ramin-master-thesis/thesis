\chapter{Future work and outlook}
\label{chap:future-work}
The subject itself has progressively gained importance within the broad field of computer science. Still, based on the research findings and limitations, further research needs to be conducted to develop the literature. Three ideas have been proposed which build the basic blocks of the future outlook:

\begin{enumerate}
    \item Load Balancer proposed in section 3.6 provides a good starting point for discussion and further research. After implementing the load balancer, the overall performance should be evaluated. Further research can be undertaken by measuring the overhead added by the load balancer during partitioning.
    \item The second idea in section 3.7 presents a new use case, where the primary research approach of this work can be adopted. Future investigations are necessary to validate the concept described.
    \item The crawled dataset from Twitter misses feedback data on the produced recommendation. Having no feedback on the generated recommendations makes it challenging to distinguish if the yield items are relevant or not. The relevancy of an item affects the overall evaluation result. Therefore, during the assessment, this research assumes that all the results in the baseline were relevant items. This assumption limits the evaluation. Future studies can develop and study other datasets with feedback or investigate new evaluation approaches like user testing to evaluate such systems. Additionally, tools or other approaches can be used to label the dataset.Another possibility to gather feedback is through user testing. The Web UI Dashboard introduced in section 3.8 is a good starting point for further investigations and designing user testing to evaluate the recommendation quality.
\end{enumerate}


