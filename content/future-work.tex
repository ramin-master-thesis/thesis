\chapter{Future work and outlook}
\label{chap:future-work}
This chapter describes the potential future work and outlook for the research done in the proposed thesis. Throughout this, a number of recommendations for future research are given. Furthermore, the main chapter introduced three ideas, which build the basic blocks of the future work and outlook chapter. 


The first idea is the \emph{Load Balancer} proposed in section \ref{sec:load-balancing}. The load balancer provides a good starting point for discussion and further research. First, the load balancer needs to be implemented. Then the overall performance of the load balancer needs to be evaluated. Finally, measuring the overhead added by the load balancer during partitioning is a further research study.


The second idea in section \ref{sec:second-usecase} presents a new use case, where the main research idea of this work can be adopted and used. Future investigations are necessary to validate the concept described.


The crawled dataset from Twitter misses feedback data on the produced recommendation. Having no feedback on the generated recommendations, makes it challanging to distinguesh if the yeild items are relevant or not. The relevancy of an item affects the overall evaluation result. Therefore, during the assessment, this work assumed that all the results in the baseline were relevant items. This assumption makes the evaluation strict. Future research should develop and study other datasets with feedback or investigate new evaluation approaches like user testing to evaluate such systems. Besides, tools or other approaches can be used to lable the dataset.


Another possibility to gather feedback is through user testing. The \emph{Web UI Dashboard} introduced in section \ref{sec:web-ui} is a good starting point for further investigations and designing user testing to evaluate the recommendation quality.


