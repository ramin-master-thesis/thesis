\section{User manual}
\label{sec:user-guide}
\begin{itemize}
    \item The proposed work is simulated using Python
    \item The repository can be found at \footnote{\url{https://github.com/ramin-master-thesis/salsa}}
    \item In this section, I go through step by step through the framework and how to set up the project
    \item First, I explain prerequisites needed and then explain how to use the partitioner and start the webserver to load the data and use the recommendation algorithm.
\end{itemize}

\subsection{Prerequisites}
\label{subsec:prerequisites}
\begin{itemize}
    \item Clone the repository on your computer
    \item Create a python virtual environment and run pip install -r requirements.txt. This will install all the necessary dependencies.
    \item Move the dataset to the \emph{data} forlder
    \item You can find the datset from here \todo{Add dataset URL}
    \item The dataset should be in a tsv or pandas parquet format
\end{itemize}

\subsection{Framework structure}
\label{subsec:framework-structure}
\begin{itemize}
    \item The project is devided into multiple modules. In this section, I explain each module.
    \item algorithm module: This module is responsible for the recommendation algorithm. The algorithm used here is SALSA \cite{lempel2001salsa}
    \item data folder: This folder contains the necessary data. The main dataset should be at the root of this folder. Partitions get seved in this folder. In other words, multiple folders get created whenever the partitioner creates the partition. As an example, if the user creates two partitions using the murmur2 algorithm the folder structure would look like this.
    \dirtree{%
    .1 data.
    .2 murmur2.
    .3 partition\_0.
    .4 left\_index.gzip.
    .4 right\_index.gzip.
    .4 content\_index.gzip.
    .3 paritition\_1.
    .4 left\_index.gzip.
    .4 right\_index.gzip.
    .4 content\_index.gzip.
    }
    \item in the data folder there also a \emph{StarSpace\_data} folder. This folder should contain the training dataset. Moreover, this folder contains all the trained StarSpace models inside of the folder called \emph{model}.
    \item Graph Module: This module is provides access to the bipartite graph or the singel content graph.
    \item Indexer: This module is responsible for indexing the data. The SALSA algorithm work primariy on a bipartite graph. Threfore the data needs to get indexed from both side.
    \item Partitioner: This module partitions the data and calls the indexer to index the data. This module contains a submodule called \emph{hash\_functions}. This submodule contains all the hash functions implementation needed to partition the data. For more implementation detail please refere to \ref{sec:partitioning} section.
    \item Server: This module contains all the code necessary to load the partitions and start the webserver. This module includes all the endpoints as well.
    \item Star\_Space: This module is responsible to train, generate data, and calculate the projection matrix for the StarSpace model.
\end{itemize}

\subsection{Usage}
\label{subsec:usage}
\begin{itemize}
    \item This section primarily focuses on how to use the framework
    \item The frist subsection shows the usage of the partitioner and in the next subsection I explain how to load the indexes start the webserver for each partition 
\end{itemize}

\subsubsection{Partitioner and Indexer}
\label{subsubsec:partitioner-indexer}
\begin{itemize}
    \item As explained in the architecture section \ref{sec:partitioning} this part partitions and creates the birpartite graph. The partitioner has a commandline interface.
    \item to start the partitioner use: 
    python3 -m partitioner.main [OPTIONS] COMMAND [ARGS]...
    \item The options that can be used here are listed in the table \ref{tab:options-partitioner}
    \item The commands here are the partitionign techniques to use to shard the data in multiple partitions. These commands are:
    single, modulo, murmur2, star-space.
    \item The above command take also some arguments. star-space, murmur2 accepts an argument \emph{-n} or \emph{--parititon-number}. This arguments determines in how many partitions the data should get sharded. The default value is two.
    \item The star-space command takes an extra argument \emph{-m} or \emph{model-folder}. This argument is the model folder name of StarSpace. This model folder should exist under ./data/StarSpace\_data/models/<Folder\_name>. This argument is required.
\end{itemize}

\begin{table}[!h]
	\centering
	\caption{Options of the partitioner}
	\label{tab:options-partitioner}
	\begin{tabular}{|l|m{0.4\linewidth}|m{0.2\linewidth}|}
		\hline
		\textbf{Option} & \textbf{Description} & \textbf{Default Value} \\
		\hline
		-f or --path-to-file & path to the tsv data file & /data/tweets-dump.tsv \\
		\hline
		--content-index/--no-content-index & A flag dedicating if the content index should be generated or not & false \\
		\hline
	\end{tabular}
\end{table}
