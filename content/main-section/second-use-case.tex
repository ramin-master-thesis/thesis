\section{Second use-case: Tag Recommendation for Stack Overflow Questions}
\label{sec:second-usecase}
The approach introduced in this thesis can be generalized and used in other use cases as well. In the following we will present a use case where the partitioner and the StarSpace model are used to distribute document data on multiple workers.


\emph{Stack Overflow}\footnote{\url{https://stackoverflow.com/}} is a website in which developers either ask or answer informatic-related questions. Each question contains one or multiple \emph{tags}. A tag is a phrase that expresses the subject of a question. Tags let professionals connect with questions they can answer by putting topics into distinct, well-defined categories.


The problem is defined as follows: \emph{How can we predict and recommend proper tags for a newly asked question?} The data dump provided by stack overflow\footnote{\url{https://archive.org/download/stackexchange}} shows that each question interacts with at least one tag. Therefore, we can build a question-tag bipartite graph from the data and distribute it based on the content of questions using the techniques proposed in the previous sections.


During partition time, each question will land on a worker along with its embedding vector. Whenever a new question is asked, we can calculate its partition-ID through its embedding vector. Next, we can find the "nearest" question to the newly asked question using the partition-ID and embedding vector. This question would be the initial vertex to start the SALSA random walk. The result of the random walk would be a ranked list of tag suggestions for the new question. The user can decide which tags to keep and which to decline.
