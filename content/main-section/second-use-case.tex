\section{Other Use Cases}
\label{sec:other-use-cases}
This thesis introduces an adaptive model-based partitioning module. This module learns the dependency between the data through the algorithm's output and delivers a more optimal partitioning of the data. Moreover, the system, which is distributed with the partitioner, is considered a "black box." So the main algorithm of the system is not changed at all. To investigate the proposed partitioning method, this work employs GraphJet, a real-time recommendation system. This section introduces two other use cases for further reach, where the partitioning idea of this thesis can be applied to.

\subsection{Telemed5000 Distributed Patients Recommendation System}
\label{subsec:telemed5000}
Telemed5000 intends to create an intelligent system for telemedical supervision of thousands of patients with heart failure, in a collaboration between the Charité, Hasso Plattner Institute, Getemed, Fraunhofer IAIS, and Synios. The preceding study \cite{koehlerEfficacyTelemedicalInterventional2018} proves that telemedical care can prolong patients' lives and spend fewer days in the hospital. The researchers in \cite{gontarskaPredictingMedicalInterventions2021} propose a decision system that employs a machine learning model to determine the risk score of a patient based on their vital parameters. Estimating a risk score allows medical practitioners to sort all cases every day and focus their limited resources on the most severe cases. 


An idea of a patient recommendation system can be applied whenever a doctor is examining a patient. The doctor gets a list of a suggested patients with the same medical situation. Although the recommendation algorithm is a field of further study, the output of this algorithm can be employed to train the partitioner's model in order to handle the growing amount of patient data. The recommendation algorithm and the distribution of the system are an issue for future research to explore.


\subsection{Tag Recommendation for Stack Overflow Questions}
\label{subsec:stackoverflow-tag-recommendation}
The approach introduced in this thesis can be generalized and used in other use cases as well. In the following we will present a use case where the partitioner and the StarSpace model are used to distribute document data on multiple workers.


\emph{Stack Overflow}\footnote{\url{https://stackoverflow.com/}} is a website in which developers either ask or answer informatic-related questions. Each question contains one or multiple \emph{tags}. A tag is a phrase that expresses the subject of a question. Tags let professionals connect with questions they can answer by putting topics into distinct, well-defined categories.


The problem is defined as follows: \emph{How can we predict and recommend proper tags for a newly asked question?} The data dump provided by stack overflow\footnote{\url{https://archive.org/download/stackexchange}} shows that each question interacts with at least one tag. Therefore, we can build a question-tag bipartite graph from the data and distribute it based on the content of questions using the techniques proposed in the previous sections.


During partition time, each question will land on a worker along with its embedding vector. Whenever a new question is asked, we can calculate its partition-ID through its embedding vector. Next, we can find the "nearest" question to the newly asked question using the partition-ID and embedding vector. This question would be the initial vertex to start the SALSA random walk. The result of the random walk would be a ranked list of tag suggestions for the new question. The user can decide which tags to keep and which to decline.
