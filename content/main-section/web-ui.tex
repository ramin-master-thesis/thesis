\section{Web UI dashboard to observe recommendations}
\label{sec:web-ui}
One way to assess the recommendation quality is to use mathematical approaches, described in \ref{subsec:evaluation-metrics}. These evaluation metrics can determine if a ranked list contains "good enough" recommendations for the user by comparing the observed rankings with the golden standard. In practice, the assessed baseline contains implicit feedback. Examples for implicit feedback are clicks, watched movies, played songs, or interaction with tweets. Positive feedback defines relevant recommendations for a user, and negative feedback labels a recommendation as not relevant. Labeling the baseline data provides a more accurate calculation of the recommendation evaluation metrics.


In studies like \cite{eksombatchaiPixieSystemRecommending2018} and \cite{goelWhoToFollowSystemTwitter2015} the authors evaluate the recommendation quality through A/B testing. A random set of users experience the recommendations generated by the new recommender system (group A), while another group of users experiences the recommendations of the old recommendations system (group B). The authors then measure the user engagement lift in clicking or liking the recommendations. 


While at Twitter and Pinterest, the authors can run online tests, it was not the main focus of this work to run a massive online test to measure the users' engagement. Furthermore, the dataset crawled (see section \ref{sec:data-crawling-strategy}) to evaluate the system misses the user feedback on their recommendations. The missing feedback data makes it harder to generate a baseline containing related and not related data. This study proposes a Web UI to see user-generated recommendations by different partitioning methods of each partition. Moreover, it provides the user's interests and latest interaction and its tweeter timeline embedding. The UI can give the users an overview of whether the recommendations returned from other partitions are "good enough" or not.


Figure \ref{fig:web-ui-intro} shows each part of the web-UI. The user first selects a user-ID, and all the recommendations will be available for the given partitions and partitioning methods. The web UI also indicates an embedding of the user's Twitter timeline so that users can see the actual activities of the user. 

\begin{figure}[!htb]
	\centering
	\includegraphics[width=1\textwidth]{images/web-ui-intro}
	\caption{Elements of the Web UI}
	\label{fig:web-ui-intro}
\end{figure}

Figure \ref{fig:web-ui-example-1} and \ref{fig:web-ui-example-2} demonstrait the expanded version of the Web UI for a given user.

\begin{figure}[!htb]
	\centering
	\includegraphics[width=1\textwidth]{images/web-ui-1}
	\caption{The Web UI for a given user. The upper part of the UI provides an embedding of the user's Tweeter timeline. Followed by the list of tweets that the user interacted with.}
	\label{fig:web-ui-example-1}
\end{figure}

\begin{figure}[!htb]
	\centering
	\includegraphics[width=1\textwidth]{images/web-ui-2}
	\caption{List of recommendations generated on different partitions. The blue circled dots denote the hit number for each item. The degree number (i.e., number of user's interactions) is also presented.}
	\label{fig:web-ui-example-2}
\end{figure}
