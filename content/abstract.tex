% => Wenn die Arbeit auf Deutsch verfasst wurde, verlangt das Studienreferat KEINEN englischen Abstract

% % englischer Abstract
\null\vfil
\begin{otherlanguage}{english}
\begin{center}\textsf{\textbf{\abstractname}}\end{center}

\noindent
%Topic
The distribution of a system in a streaming environment that runs on a single instance can be parallelized in two ways: First, distributing its algorithm and dividing the tasks into multiple map operations, and reducing the results. However, this approach needs adjustment of the algorithm, plus it is cost-inefficient due to frequent network calls between the nodes. The second method distributes the incoming stream of (document) data on multiple instances without changing the main algorithm. Whereas these data are highly connected and dependent on each other. So randomly spreading the data on the instances leads to poor results.

%Purpose
This thesis aims to partition the data with respect to their dependency on each other to deliver "good enough" results. This work proposes a partitioning module based on a machine learning model, which learns the dependency between the data only by observing the inputs and outputs of the system. Therefore, the system itself is assumed as a "black box," which makes this partitioning module adaptive for different use cases. The trained model sends the data to workers based on their topic relevancy for incoming write requests. Each worker runs the algorithm in complete isolation, and the gathered results are integrated into a single representation. The ultimate goal is to produce a single representation that does not deviate from the single instance result.

%usecase
GraphJet, a graph-based recommendation system, has been chosen as a case study to demonstrate the proposed partitioning technique. With the help of an evaluation suite and an assessment pipeline, various experiments have been designed to compare the recommendation results of a single instance against the multi-partition model of GraphJet.


%resutls
The experiments results indicate that the introduced method addresses better results compared to a random partitioning method. Still, comparing the results with single instance results shows differences between the generated recommendations for a user. However, we have not found enough evidence to conclude that these different recommended items necessarily mean that the system's partitioning with the introduced method will deliver poor results.


\end{otherlanguage}
\vfil\null
\newpage
% => Wenn die Arbeit auf Englisch verfasst wurde, verlangt das Studienreferat einen englischen UND deutschen Abstract (der dt. Abstract kann dann ggf. auch ans Ende der Arbeit)

% deutsche Zusammenfassung
\null\vfil
\begin{otherlanguage}{ngerman}
\begin{center}\textsf{\textbf{\abstractname}}\end{center}

\noindent 
%Thema
Die Verteilung eines Systems in einer Streaming-Umgebung, das auf einer einzigen Instanz läuft, kann auf zwei Arten parallelisiert werden: Erstens, indem man seinen Algorithmus verteilt und die Aufgaben in mehrere Kartenoperationen aufteilt und die Ergebnisse reduziert. Dieser Ansatz erfordert jedoch eine Anpassung des Algorithmus und ist aufgrund der häufigen Netzwerkanrufe zwischen den Knoten kostenineffizient. Bei der zweiten Methode wird der eingehende Strom von (Dokumenten-)Daten auf mehrere Instanzen verteilt, ohne den Hauptalgorithmus zu ändern. Diese Daten sind jedoch stark miteinander verbunden und voneinander abhängig. Eine willkürliche Verteilung der Daten auf die Instanzen führt daher zu schlechten Ergebnissen.

%Zweck
Diese Arbeit zielt darauf ab, die Daten im Hinblick auf ihre Abhängigkeit voneinander zu partitionieren, um "ausreichend gute" Ergebnisse zu liefern. In dieser Arbeit wird ein Partitionierungsmodul vorgeschlagen, das auf einem maschinellen Lernmodell basiert, das die Abhängigkeit zwischen den Daten nur durch Beobachtung der Ein- und Ausgaben des Systems lernt. Daher wird das System selbst als "Black Box" angenommen, was dieses Partitionierungsmodul für verschiedene Anwendungsfälle anpassungsfähig macht. Das trainierte Modell sendet die Daten auf der Grundlage ihrer thematischen Relevanz für eingehende Schreibanforderungen an die Arbeiter. Jeder Arbeiter führt den Algorithmus völlig isoliert aus, und die gesammelten Ergebnisse werden in eine einzige Darstellung integriert. Das ultimative Ziel ist es, eine einzige Darstellung zu erzeugen, die nicht von dem Ergebnis der einzelnen Instanz abweicht.

%Anwendungsfall
GraphJet, ein graphbasiertes Empfehlungssystem, wurde als Fallstudie ausgewählt, um die vorgeschlagene Partitionierungstechnik zu demonstrieren. Mit Hilfe einer Evaluierungssuite und einer Bewertungspipeline wurden verschiedene Experimente entworfen, um die Empfehlungsergebnisse einer einzelnen Instanz mit dem Multi-Partitionsmodell von GraphJet zu vergleichen.


%Ergebnisse
Die Ergebnisse der Experimente zeigen, dass die vorgestellte Methode im Vergleich zu einer zufälligen Partitionierungsmethode bessere Ergebnisse liefert. Dennoch zeigt der Vergleich der Ergebnisse mit den Ergebnissen einer einzelnen Instanz Unterschiede zwischen den generierten Empfehlungen für einen Nutzer. Wir haben jedoch nicht genügend Beweise gefunden, um daraus zu schließen, dass diese unterschiedlichen Empfehlungen zwangsläufig bedeuten, dass die Partitionierung des Systems mit der vorgestellten Methode schlechte Ergebnisse liefert.

\end{otherlanguage}
\vfil\null



