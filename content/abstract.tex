% => Wenn die Arbeit auf Deutsch verfasst wurde, verlangt das Studienreferat KEINEN englischen Abstract

% % englischer Abstract
\null\vfil
\begin{otherlanguage}{english}
\begin{center}\textsf{\textbf{\abstractname}}\end{center}

\noindent This work proposes a novel approach in partitioning text data based on their contents using an embedding model called StarSpace. This approach can be generalized to systems that use text data. With this partitioning technique, the text documents similar to each other are classified together and land on the same partition. To demonstrate and evaluate the researched approach of this thesis, a particular use case based on the work developed by Twitter, called GraphJet, is leveraged. GraphJet maintains a bipartite user-tweet graph on a single machine and runs a random-walk algorithm called SALSA to generate real-time user recommendations. Furthermore, this research implements an evaluation suite along with an assessment pipeline for comparing the recommendation results of a single instance against the multi-partition instance of GraphJet.

\end{otherlanguage}
\vfil\null


% => Wenn die Arbeit auf Englisch verfasst wurde, verlangt das Studienreferat einen englischen UND deutschen Abstract (der dt. Abstract kann dann ggf. auch ans Ende der Arbeit)

% deutsche Zusammenfassung
\null\vfil
\begin{otherlanguage}{ngerman}
\begin{center}\textsf{\textbf{\abstractname}}\end{center}

\noindent In dieser Arbeit wird ein neuartiger Ansatz zur Partitionierung von Textdaten auf der Grundlage ihres Inhalts unter Verwendung eines Einbettungsmodells namens StarSpace vorgeschlagen. Dieser Ansatz kann für Systeme, die Textdaten verwenden, verallgemeinert werden. Mit dieser Partitionierungstechnik werden die einander ähnlichen Textdokumente gemeinsam klassifiziert und landen auf derselben Partition. Zur Demonstration und Bewertung des in dieser Arbeit erforschten Ansatzes wird ein spezieller Anwendungsfall auf der Grundlage der von Twitter entwickelten Arbeit, GraphJet, verwendet. GraphJet verwaltet einen bipartiten Nutzer-Tweet-Graphen auf einer einzigen Maschine und führt einen Random-Walk-Algorithmus namens SALSA aus, um Nutzerempfehlungen in Echtzeit zu generieren. Darüber hinaus befasst sich diese Forschungsarbeit mit einer Evaluierungssuite und einer Bewertungspipeline zum Vergleich der Empfehlungsergebnisse einer einzelnen Instanz mit der Multi-Partition-Instanz von GraphJet.



\end{otherlanguage}
\vfil\null



