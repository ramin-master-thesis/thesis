% => Wenn die Arbeit auf Deutsch verfasst wurde, verlangt das Studienreferat KEINEN englischen Abstract

% % englischer Abstract
\null\vfil
\begin{otherlanguage}{english}
\begin{center}\textsf{\textbf{\abstractname}}\end{center}

\noindent This work proposes a novel approach in partitioning document data based on their contents using an embedding model called StarSpace. Documents similar to each other land on the same partition. The particular use case to demonstrate and evaluate this approach is based on the work developed by Twitter called GraphJet. GraphJet maintains a bipartite user-tweet graph and runs a random walk algorithm called SALSA to generate real-time user recommendations.

\end{otherlanguage}
\vfil\null


% => Wenn die Arbeit auf Englisch verfasst wurde, verlangt das Studienreferat einen englischen UND deutschen Abstract (der dt. Abstract kann dann ggf. auch ans Ende der Arbeit)

% deutsche Zusammenfassung
\null\vfil
\begin{otherlanguage}{ngerman}
\begin{center}\textsf{\textbf{\abstractname}}\end{center}

\noindent In dieser Arbeit wird ein neuartiger Ansatz zur Partitionierung von Dokumentendaten auf der Grundlage ihres Inhalts unter Verwendung eines Einbettungsmodells namens StarSpace vorgeschlagen. Dokumente, die einander ähnlich sind, landen in der gleichen Partition. Der spezielle Anwendungsfall zur Demonstration und Bewertung dieses Ansatzes basiert auf der von Twitter entwickelten Arbeit namens GraphJet. GraphJet verwaltet einen bipartiten Nutzer-Tweet-Graphen und führt einen Random-Walk-Algorithmus namens SALSA aus, um Nutzerempfehlungen in Echtzeit zu generieren.

\end{otherlanguage}
\vfil\null



