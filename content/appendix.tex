\chapter{\appendixname}
% \section*{Murmur2 Node Distribution Over two Partitions}


\begin{figure}[!htb]
    \centering
    \begin{subfigure}{\textwidth}
        \centering
        \begin{tikzpicture}
	\begin{axis}[,
			boxplot/draw direction=y,
			x axis line style={opacity=0},
			axis x line*=bottom,
			xtick={1,2,3},
			xticklabels={Single, Partition 0, Partition 1},
			enlarge x limits=0.1,
			xtick distance=2,
			ylabel=Seconds,
			width = 12cm, height=8cm,
		]
		\addplot+ [boxplot]
		table [y index=0] {appendix/recommendation-latency.dat};

		\addplot+ [boxplot]
		table [y index=1] {appendix/recommendation-latency.dat};

		\addplot+ [boxplot]
		table [y index=2] {appendix/recommendation-latency.dat};
	\end{axis}
\end{tikzpicture}
        \caption{Single partition recommendation generation latency compared to two partitions}
        \label{plot:recommendation-latency-two-partitions}
    \end{subfigure}\qquad

    \begin{subfigure}{\textwidth}
        \centering
        \begin{tikzpicture}
	\begin{axis}[,
			boxplot/draw direction=y,
			x axis line style={opacity=0},
			axis x line*=bottom,
			xtick={1,2,3,4},
			xticklabels={Partition 0, Partition 1, Partition 2, Partition 3},
			enlarge x limits=0.1,
			xtick distance=2,
			ylabel=Seconds,
			height = 6cm, width = 12cm
		]

		\addplot+ [boxplot]
		table [y index=1] {appendix/recommendation-latency-four-partition.dat};

		\addplot+ [boxplot]
		table [y index=2] {appendix/recommendation-latency-four-partition.dat};

		\addplot+ [boxplot]
		table [y index=3] {appendix/recommendation-latency-four-partition.dat};

		\addplot+ [boxplot]
		table [y index=4] {appendix/recommendation-latency-four-partition.dat};
	\end{axis}
\end{tikzpicture}
        \caption{Recommendation generation latency in four partition}
        \label{plot:recommendation-latency-four-partitions}
    \end{subfigure}\qquad
    \caption{Recommendation generation latency in seconds.}
    \label{plot:recommendation-latency}
\end{figure}

\begin{figure}[!htb]
    \centering
	\begin{subfigure}{\textwidth}
		\centering
		\begin{tikzpicture}
	\begin{axis}[
			ybar,
			height=7cm,
			ymin=0,ymax=3545044,
			ylabel={\#Left Nodes},
			xlabel={Hash Function},
			ytick={0,1772522,3545044},
			symbolic x coords={Murmur2},
			scaled y ticks=false,
			yticklabel style={
				/pgf/number format/fixed,
				/pgf/number format/precision=5
			},
			xtick=data,
			bar width=30,
			nodes near coords, 
			node near coords style={
				xshift=0.05cm,
				/utils/exec={\setbox0\hbox{\pgfmathprintnumber\pgfplotspointmeta}
					\pgfmathfloattomacro{\pgfplotspointmeta}{\F}{\M}{\E}
					\pgfmathsetmacro{\myanchor}{
						ifthenelse(\M*pow(10,\E-3)*2-\the\wd0>0,"east","west")
						}
				},
				rotate=90,
				anchor=\myanchor
				},
				every node near coord/.append style={
					/pgf/number format/fixed,
					/pgf/number format/precision=5
				},
				legend style={
					at={(0.5,-0.2)},
					anchor=north,legend columns=-1
				},
				]
				\addplot 
				table [y index=1, x index=0] {appendix/murmur2-left-node-distribution.dat};
				\addplot 
				table [y index=2, x index=0] {appendix/murmur2-left-node-distribution.dat};
				\legend{Partition 0,Partition 1}
	\end{axis}
\end{tikzpicture}
		\caption{Left node (users) distribution}
		\label{plot:left-node-distribution-murmur2}
    \end{subfigure}\qquad

    \begin{subfigure}{\textwidth}
		\centering
		\begin{tikzpicture}
	\begin{axis}[
					ybar,
					height=6.5cm,
					ymin=0,ymax=3545044,
					ylabel={\#Right Nodes},
					xlabel={Hash Function},
					ytick={0,1772522,3545044},
					symbolic x coords={Murmur2},
					scaled y ticks=false,
					yticklabel style={
						/pgf/number format/fixed,
						/pgf/number format/precision=5
					},
					xtick=data,
					bar width=30,
					nodes near coords,
					every node near coord/.append style={
						xshift=0.05cm,
						rotate=90,anchor=east,
						/pgf/number format/fixed,
						/pgf/number format/precision=0
					},
					legend style={
						at={(0.5,-0.25)},
						anchor=north,legend columns=-1
					}
			]
				\addplot 
				table [y index=1, x index=0] {appendix/murmur2-right-node-distribution.dat};
				\addplot 
				table [y index=2, x index=0] {appendix/murmur2-right-node-distribution.dat};
				\legend{Partition 0,Partition 1}
	\end{axis}
\end{tikzpicture}
		\caption{Right node (tweets) distribution}
		\label{plot:right-node-distribution-murmur2}
    \end{subfigure}\qquad
    
    \caption{Node distirbution on two partitions using Murmur2 partitioning method. In (a) the sum of left nodes of two partitions is bigger than the total amount of the left nodes. This is because we replicate the left nodes on the partitions. In (b) the sum of two partitions maches exactly the total amount, since the partitioning is based on the right side verticies.}
\end{figure}

% \section*{StarSpace Node Distribution Over Two Partitions}

\begin{figure}[!htb]
	\centering
	\begin{tikzpicture}
	\begin{axis}[
			xbar,
			xmin=0,xmax=3545044,
			height=19cm, width=13cm,
			enlarge y limits=0.05,
			xlabel={\#Left Nodes},
			ylabel={Model Nr.},
			xtick={0,1772522,3545044},
			scaled x ticks=false,
			xticklabel style={
				/pgf/number format/fixed,
				/pgf/number format/precision=5
			},
			ytick=data,
			nodes near coords, 
			nodes near coords align={horizontal},
			every node near coord/.append style={
				/pgf/number format/fixed,
				/pgf/number format/precision=5
			},
			legend style={at={(0.5,-0.10)},
				anchor=north,legend columns=-1},
		]
		\addplot
		table [y index=0, x index=1] {appendix/hyperparameter-left-node-distribution.dat};
		\addplot 
		table [y index=0, x index=2] {appendix/hyperparameter-left-node-distribution.dat};
		\legend{Partition 0,Partition 1}
	\end{axis}
\end{tikzpicture}
	\caption{Left node (users) distribution of test dataset; segmented with different trained models over two partitions}
	\label{plot:left-node-distribution}
\end{figure}

\begin{figure}[!htb]
	\centering
	\begin{tikzpicture}
	\begin{axis}[
			xbar,
			xmin=0,xmax=3439934,
			height=19cm, width=13cm,
			enlarge y limits=0.05,
			xlabel={\#Right Nodes},
			ylabel={Model Nr.},
			xtick={0,1719967,3439934},
			scaled x ticks=false,
			xticklabel style={
				/pgf/number format/fixed,
				/pgf/number format/precision=5
			},
			ytick=data,
			nodes near coords, 
			nodes near coords align={horizontal},
			every node near coord/.append style={
				/pgf/number format/fixed,
				/pgf/number format/precision=5
			},
			legend style={at={(0.5,-0.10)},
				anchor=north,legend columns=-1},
		]
		\addplot
		table [y index=0, x index=1] {appendix/hyperparameter-right-node-distribution.dat};
		\addplot 
		table [y index=0, x index=2] {appendix/hyperparameter-right-node-distribution.dat};
		\legend{Partition 0,Partition 1}
	\end{axis}
\end{tikzpicture}
	\caption{Right node (tweets) distribution of test dataset; segmented with different trained models over two partitions}
	\label{plot:right-node-distribution}
\end{figure}


% \section*{Rank-Biased Overlap}


\begin{figure}[!htb]
	\centering
	\begin{subfigure}[b]{0.5\linewidth}
	  \centering
	  \begin{tikzpicture}
	\begin{axis}[
			xlabel=K,
			ylabel=RBO,
			width=7cm,height=7cm,
			xmin=0,xmax=10.5,
			xtick distance=1,
			ymin=0,ymax=1,
			legend style={
                at={(0.5,-0.2)},
			    anchor=north,legend columns=-1
            },
            legend columns=3,
            transpose legend
            ]
				    
		\addplot[
			color=blue,
            mark=*,
            mark options={solid},
		]
		table [x=k, y index=0] {appendix/2-partitions.dat};
        \addlegendentry{Single Partition}

        \addplot[
			color=red,
            mark=x,
            mark options={solid},
		]
		table [x=k, y index=1] {appendix/2-partitions.dat};
        \addlegendentry{Murmur2 Union Results}

        \addplot[
			color=brown,
            mark=+,
            mark options={solid},
		]
		table [x=k, y index=2] {appendix/2-partitions.dat};		    
		\addlegendentry{StarSpace Most Interactions}
	\end{axis}
\end{tikzpicture}
	  \caption{2 Partitions} 
	  \label{fig:RBO-horizontall-scaling-2-partitions-a} 
	  \vspace{1cm}
	\end{subfigure}%% 
	\begin{subfigure}[b]{0.5\linewidth}
	  \centering
	  \begin{tikzpicture}
	\begin{axis}[
			xlabel=K,
			ylabel=RBO,
			width=7cm,height=7cm,
			xmin=0,xmax=10.5,
			xtick distance=1,
			ymin=0,ymax=1,
			legend style={
                at={(0.5,-0.2)},
			    anchor=north,legend columns=-1
            },
            legend columns=3,
            transpose legend
            ]
				    
		\addplot[
			color=blue,
            mark=*,
            mark options={solid},
		]
		table [x=k, y index=0] {appendix/4-partitions.dat};
        \addlegendentry{Single Partition}

        \addplot[
			color=red,
            mark=x,
            mark options={solid},
		]
		table [x=k, y index=1] {appendix/4-partitions.dat};
        \addlegendentry{Murmur2 Union Results}

        \addplot[
			color=brown,
            mark=+,
            mark options={solid},
		]
		table [x=k, y index=2] {appendix/4-partitions.dat};		    
		\addlegendentry{StarSpace Most Interactions}
	\end{axis}
\end{tikzpicture}
	  \caption{4 Partitions} 
	  \label{fig:RBO-horizontall-scaling-4-partitions-b} 
	  \vspace{1cm}
	\end{subfigure} 
	\begin{subfigure}[b]{0.5\linewidth}
	  \centering
	  \begin{tikzpicture}
	\begin{axis}[
			xlabel=K,
			ylabel=RBO,
			width=7cm,height=7cm,
			xmin=0,xmax=10.5,
			xtick distance=1,
			ymin=0,ymax=1,
			legend style={
                at={(0.5,-0.2)},
			    anchor=north,legend columns=-1
            },
            legend columns=3,
            transpose legend
            ]
				    
		\addplot[
			color=blue,
            mark=*,
            mark options={solid},
		]
		table [x=k, y index=0] {appendix/8-partitions.dat};
        \addlegendentry{Single Partition}

        \addplot[
			color=red,
            mark=x,
            mark options={solid},
		]
		table [x=k, y index=1] {appendix/8-partitions.dat};
        \addlegendentry{Murmur2 Union Results}

        \addplot[
			color=brown,
            mark=+,
            mark options={solid},
		]
		table [x=k, y index=2] {appendix/8-partitions.dat};		    
		\addlegendentry{StarSpace Most Interactions}
	\end{axis}
\end{tikzpicture}
	  \caption{8 Partitions} 
	  \label{fig:RBO-horizontall-scaling-8-partitions-c} 
	\end{subfigure}%%
	\begin{subfigure}[b]{0.5\linewidth}
	  \centering
	  \begin{tikzpicture}
	\begin{axis}[
			xlabel=K,
			ylabel=RBO,
			width=7cm,height=7cm,
			xmin=0,xmax=10.5,
			xtick distance=1,
			ymin=0,ymax=1,
			legend style={
                at={(0.5,-0.2)},
			    anchor=north,legend columns=-1
            },
            legend columns=3,
            transpose legend
            ]
				    
		\addplot[
			color=blue,
            mark=*,
            mark options={solid},
		]
		table [x=k, y index=0] {appendix/16-partitions.dat};
        \addlegendentry{Single Partition}

        \addplot[
			color=red,
            mark=x,
            mark options={solid},
		]
		table [x=k, y index=1] {appendix/16-partitions.dat};
        \addlegendentry{Murmur2 Union Results}

        \addplot[
			color=brown,
            mark=+,
            mark options={solid},
		]
		table [x=k, y index=2] {appendix/16-partitions.dat};		    
		\addlegendentry{StarSpace Most Interactions}
	\end{axis}
\end{tikzpicture}
	  \caption{16 Partitions} 
	  \label{fig:RBO-horizontall-scaling-16-partitions-d} 
	\end{subfigure} 
	\caption{Assessing the Recommendation quality with RBO when scaling horizontally. The \emph{p} parameter is set to 0.9, which gives the first ten results 86\% of the weight in the similarity comparison.}
	\label{fig:RBO-horizontall-scaling} 
\end{figure}
