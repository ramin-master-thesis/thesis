\chapter{\appendixname}
\todo{WIP}
\section*{StarSpace Node distribution over two partitions}

\begin{figure}[!hbt]
	\centering
	\begin{tikzpicture}
	\begin{axis}[
			xbar,
			xmin=0,xmax=3545044,
			height=19cm, width=13cm,
			enlarge y limits=0.05,
			xlabel={\#Left Nodes},
			ylabel={Model Nr.},
			xtick={0,1772522,3545044},
			scaled x ticks=false,
			xticklabel style={
				/pgf/number format/fixed,
				/pgf/number format/precision=5
			},
			ytick=data,
			nodes near coords, 
			nodes near coords align={horizontal},
			every node near coord/.append style={
				/pgf/number format/fixed,
				/pgf/number format/precision=5
			},
			legend style={at={(0.5,-0.10)},
				anchor=north,legend columns=-1},
		]
		\addplot
		table [y index=0, x index=1] {appendix/hyperparameter-left-node-distribution.dat};
		\addplot 
		table [y index=0, x index=2] {appendix/hyperparameter-left-node-distribution.dat};
		\legend{Partition 0,Partition 1}
	\end{axis}
\end{tikzpicture}
	\caption{Left node distribution of test dataset; sharded with different trained models over two partitions}
	\label{plot:left-node-distribution}
\end{figure}

\begin{figure}[!hbt]
	\centering
	\begin{tikzpicture}
	\begin{axis}[
			xbar,
			xmin=0,xmax=3439934,
			height=19cm, width=13cm,
			enlarge y limits=0.05,
			xlabel={\#Right Nodes},
			ylabel={Model Nr.},
			xtick={0,1719967,3439934},
			scaled x ticks=false,
			xticklabel style={
				/pgf/number format/fixed,
				/pgf/number format/precision=5
			},
			ytick=data,
			nodes near coords, 
			nodes near coords align={horizontal},
			every node near coord/.append style={
				/pgf/number format/fixed,
				/pgf/number format/precision=5
			},
			legend style={at={(0.5,-0.10)},
				anchor=north,legend columns=-1},
		]
		\addplot
		table [y index=0, x index=1] {appendix/hyperparameter-right-node-distribution.dat};
		\addplot 
		table [y index=0, x index=2] {appendix/hyperparameter-right-node-distribution.dat};
		\legend{Partition 0,Partition 1}
	\end{axis}
\end{tikzpicture}
	\caption{Right node distribution of test dataset; sharded with different trained models over two partitions}
	\label{plot:right-node-distribution}
\end{figure}

\section*{Murmur2 Node distribution over two partitions}
\begin{figure}[!hbt]
    \centering
	\begin{subfigure}{\textwidth}
		\centering
		\begin{tikzpicture}
	\begin{axis}[
			ybar,
			height=7cm,
			ymin=0,ymax=3545044,
			ylabel={\#Left Nodes},
			xlabel={Hash Function},
			ytick={0,1772522,3545044},
			symbolic x coords={Murmur2},
			scaled y ticks=false,
			yticklabel style={
				/pgf/number format/fixed,
				/pgf/number format/precision=5
			},
			xtick=data,
			bar width=30,
			nodes near coords, 
			node near coords style={
				xshift=0.05cm,
				/utils/exec={\setbox0\hbox{\pgfmathprintnumber\pgfplotspointmeta}
					\pgfmathfloattomacro{\pgfplotspointmeta}{\F}{\M}{\E}
					\pgfmathsetmacro{\myanchor}{
						ifthenelse(\M*pow(10,\E-3)*2-\the\wd0>0,"east","west")
						}
				},
				rotate=90,
				anchor=\myanchor
				},
				every node near coord/.append style={
					/pgf/number format/fixed,
					/pgf/number format/precision=5
				},
				legend style={
					at={(0.5,-0.2)},
					anchor=north,legend columns=-1
				},
				]
				\addplot 
				table [y index=1, x index=0] {appendix/murmur2-left-node-distribution.dat};
				\addplot 
				table [y index=2, x index=0] {appendix/murmur2-left-node-distribution.dat};
				\legend{Partition 0,Partition 1}
	\end{axis}
\end{tikzpicture}
		\caption{Left node distribution of test dataset.}
		\label{plot:left-node-distribution-murmur2}
    \end{subfigure}\qquad

    \begin{subfigure}{\textwidth}
		\centering
		\begin{tikzpicture}
	\begin{axis}[
					ybar,
					height=6.5cm,
					ymin=0,ymax=3545044,
					ylabel={\#Right Nodes},
					xlabel={Hash Function},
					ytick={0,1772522,3545044},
					symbolic x coords={Murmur2},
					scaled y ticks=false,
					yticklabel style={
						/pgf/number format/fixed,
						/pgf/number format/precision=5
					},
					xtick=data,
					bar width=30,
					nodes near coords,
					every node near coord/.append style={
						xshift=0.05cm,
						rotate=90,anchor=east,
						/pgf/number format/fixed,
						/pgf/number format/precision=0
					},
					legend style={
						at={(0.5,-0.25)},
						anchor=north,legend columns=-1
					}
			]
				\addplot 
				table [y index=1, x index=0] {appendix/murmur2-right-node-distribution.dat};
				\addplot 
				table [y index=2, x index=0] {appendix/murmur2-right-node-distribution.dat};
				\legend{Partition 0,Partition 1}
	\end{axis}
\end{tikzpicture}
		\caption{Right node distribution of test dataset.}
		\label{plot:right-node-distribution-murmur2}
    \end{subfigure}\qquad
    
    \caption{Node distirbution of the test dataset. In (a) the sum of left nodes of two partitions is bigger than the total amount of the left nodes. This is because we replicate the left nodes on the partitions. In (b) the sum of two partitions maches exactly the total amount, since the partitioning is based on the right side verticies.}
\end{figure}