\section{Recommendation Quality}
\label{sec:recommendation-quality}
This section will go through the experiments to evaluate the impact of the partitioning over the recommendations. The first part of this section determines the experiment setup and different parameters values used to build the scenarios. The following sections describe and discuss the results of different techniques.


\subsection{Experiment setup}
\label{subsec:experiment-setup}
As explained in section \ref{sec:evaluation-pipeline} they are parameters that we can inject into the pipeline to create different scenarios. Choosing different partitioning methods, amount of partitions, data fusion method, evaluation metric, number of sampled users, and the frequency of repeated recommendation for the baseline.


\subsection{Single Machine}
\label{subsec:eval-single-machine}
The non-deterministic behavior of the random walk algorithm in generating results makes it hard to create an excellent baseline to assess the results with. To measure the reproducibility of the results, I compare the single partition result with the generated golden standard (see section \ref{subsec:baseline-generation}). The experiments sample 500 random users and generate the observed rankings on the single partition. The recommendation quality is measured using the Mean Average Precision at K (MAP@K), with K starting from 1 to 10. The experiments are done ten times each time using the same sampled users. 

The results of the experiments are plotted on the boxplot in figure \ref{plot:single-partition-boxplot}. Only the K values of eight, nine, and ten are shown to keep the diagram small and clear to read. As the chart denotes, the variation in each cut-off (i.e., K) is not significant. Although the non-deterministic random walk generates different recommendations each time for users, it does not impact the overall MAP@K value. So the possibility that the random walk generates the same results each time is still considered as high.


\begin{figure}[h!]
    \centering
    \begin{tikzpicture}
	\begin{axis}[,
			boxplot/draw direction=y,
			x axis line style={opacity=0},
			axis x line*=bottom,
			xtick={1,2,3},
			xticklabels={K = 8, K = 9, K = 10},
			xtick distance=1,
			ymin=0, ymax=0.73,
			ylabel=MAP,
			height=10cm
		]
		% \addplot+ [boxplot]
		% table [y index=0] {single-partition-users-MAP-at-Ks.dat};
		% \addplot+ [boxplot]
		% table [y index=1] {single-partition-users-MAP-at-Ks.dat};
		% \addplot+ [boxplot]
		% table [y index=2] {single-partition-users-MAP-at-Ks.dat};
		% \addplot+ [boxplot]
		% table [y index=3] {single-partition-users-MAP-at-Ks.dat};
		% \addplot+ [boxplot]
		% table [y index=4] {single-partition-users-MAP-at-Ks.dat};
		% \addplot+ [boxplot]
		% table [y index=5] {single-partition-users-MAP-at-Ks.dat};
		% \addplot+ [boxplot]
		% table [y index=6] {single-partition-users-MAP-at-Ks.dat};
		\addplot+ [boxplot]
		table [y index=7] {recommendation/single-partition-users-MAP-at-Ks.dat};
		\addplot+ [boxplot]
		table [y index=8] {recommendation/single-partition-users-MAP-at-Ks.dat};
		\addplot+ [ boxplot ] 
		table [y index=9] {recommendation/single-partition-users-MAP-at-Ks.dat};
	\end{axis}
\end{tikzpicture}
    \caption{Different MAP values over ten interations of generating recommendation with 500 users}
    \label{plot:single-partition-boxplot}
\end{figure}


The line diagram in figure \ref{plot:single-partition} shows the recommendation quality in one of the runs over the rising value of Ks. The MAP value at a cut-off of one is 90\%, indicating that the first ranked value of the 500 users is similar to the first value of the golden standard. As K increases, the ranked list gets longer, and the MAP value decreases. They are indicating that the results of lower rank values are distance from the golden standard values. The MAP@10 value reaches 63\%. This can be considered a low MAP value. But the experiments criteria are strict regarding the fact that the baseline contains only relevant items. If an observed rank is assessed with a user's baseline and does not contain the same, it will lead to low average precision and mean average precision, respectively.


\begin{figure}[h!]
    \centering
    \begin{tikzpicture}
	\begin{axis}[
			xlabel=K,
			ylabel=MAP,
			width=10cm,height=7cm,
			xmin=0,xmax=10.5,
			xtick distance=1,
			ymin=0,ymax=1,
			legend style={at={(1,1)},anchor=north east}
		]
				    
		\addplot[
			color=blue,
            mark=*,
            mark options={solid},
		]
		table [x=k, y=map] {recommendation/single-partition.dat};
				    
		\legend{Single Partition}
	\end{axis}
\end{tikzpicture}
    \caption{Evaluation of the MAP value for increasing K for 500 users}
    \label{plot:single-partition}
\end{figure}

\subsection{Multiple Machines}
\label{subsec:eval-multiple-machines}

\subsubsection{Data fusion evaluation results}
\label{subsubsec:eval-data-fusion}
The following section will compare the recommendation quality of three data fusion methods introduced in the section \ref{sec:data-fusion-approaches} for the Murmur2 and StarSpace partitioning approaches. These benchmarks will denote which data fusion approach produces the best single ranked list for a specific partitioning method. The results are shown in the diagrams of figure \ref{plot:murmur2-data-fusion} and \ref{plot:star-space-data-fusion}.


\begin{figure}[ht!]
    \centering
    \begin{subfigure}{\textwidth}
        \centering
        \begin{tikzpicture}
	\begin{axis}[
			xlabel=K,
			ylabel=MAP,
			width=10cm,height=7cm,
			xmin=0,xmax=10.5,
			xtick distance=1,
			ymin=0,ymax=1,
			legend style={at={(1,1)},anchor=north east}
		]
				    
		\addplot[
			color=blue,
            mark=*,
            mark options={solid},
		]
		table [x=k, y index=0] {recommendation/murmur2-data-fusion.dat};

        \addplot[
			color=red,
            mark=x,
            mark options={solid},
		]
		table [x=k, y index=1] {recommendation/murmur2-data-fusion.dat};

        \addplot[
			color=brown,
            mark=+,
            mark options={solid},
		]
		table [x=k, y index=2] {recommendation/murmur2-data-fusion.dat};
				    
		\legend{Union Results, Highest Hit, Most Interactions}
	\end{axis}
\end{tikzpicture}
        \caption{Different data fusion approaches using Murmur2 hash partitioning}
        \label{plot:murmur2-data-fusion}
    \end{subfigure}\qquad

    \begin{subfigure}{\textwidth}
        \centering
        \begin{tikzpicture}
	\begin{axis}[
			xlabel=K,
			ylabel=MAP,
			width=10cm,height=7cm,
			xmin=0,xmax=10.5,
			xtick distance=1,
			ymin=0,ymax=1,
			legend style={at={(1,1)},anchor=north east}
		]
				    
		\addplot[
			color=blue,
            mark=*,
            mark options={solid},
		]
		table [x=k, y index=0] {recommendation/star-space-data-fusion.dat};

        \addplot[
			color=red,
            mark=x,
            mark options={solid},
		]
		table [x=k, y index=1] {recommendation/star-space-data-fusion.dat};

        \addplot[
			color=brown,
            mark=+,
            mark options={solid},
		]
		table [x=k, y index=2] {recommendation/star-space-data-fusion.dat};
				    
		\legend{Union Results, Highest Hit, Most Interactions}
	\end{axis}
\end{tikzpicture}
        \caption{Different data fusion approaches using StarSpace partitioning}
        \label{plot:star-space-data-fusion}
    \end{subfigure}\qquad
    \caption{Comparing different data fusion approaches on two partitions for 500 sampled users. Diagram (a) shows the results of Murmur2 hash partitioning. Diagram (b) shows the results of StarSpace partitioning.}
\end{figure}


When comparing the results of the two diagrams, we can realize that the \emph{Union Results} approach for the Murmur2 hash and the \emph{Most Interactions} approach for StarSpace partitioning works well. In the chart \ref{plot:star-space-data-fusion} the reason that the results of the two other techniques for StarSpace partitioning are so low is the hit number produced by the SALSA algorithm. The two approaches use the hit number to create the single ranked list. Compared to the Murmur2 hash, the StarSpace partitioner distributes the data unbalanced, creating a smaller and bigger partition. The hit count changes with the amount of data on each partition, respectively. Observing the recommendations and hit counts on the smaller partition shows that users with fewer interactions (i.e., nodes with a small degree number) create loops in the SALSA algorithm. These loops lead the random walk to visit the same node continuously, and this will create a recommendation with a big hit number affecting the \emph{Union Results}, and \emph{Highest Hit} approaches results.


The \emph{Most Interactions} approach works well on the StarSpace partitioning method since the model is partitioning the data based on the content of the data. This approach is not dependent on the hit number anymore and takes the partition results with the highest degree a user has. Having a higher degree on a partition for a user means that the user has more interactions with documents (i.e., tweets) related to each other. On the other hand, the \emph{Most Interactions} method does not work well for the Murmur2 hash partitioning since the data is sharded randomly and not based on the user interests. The upcoming sections will use the \emph{Most Interactions} for StarSpace partitioning and \emph{Union Results} for Murmur2 hash partitioning.


\subsubsection{Hyperparameter evaluation results}
\label{subsubsec:eval-hyperparameter}
As explained in section \ref{subsec:hyperparameter-tuning}, this thesis investigates how different trained StarSpace models affect the partitioning of the data and, finally, the recommendation quality. Four main parameters were tuned during the training. These parameters can be found in the table \ref{tab:hyperparameter}. In the following table \ref{tab:models} you will find 16 models and the parameters with which they got trained.


The hyperparameter tuning benchmark was done on a two-partition setting to keep the evaluation simple. The hyperparameter tuning experiments were structured in the following steps:

\begin{enumerate}
    \item Train the StarSpace model with the training dataset and the parameters listed in each row of the table \ref{tab:models}.
    \item Calculate the projection matrix of the model (see section \ref{subsec:projection-matrix-calculation}).
    \item Load the model in memory and partition the test dataset on two partitions.
    \item Load sampled users (500, 2000) and generated the recommendations for each of them.
    \item Use the \emph{Most Interactions} data fusion approach to generate a single ranked list of recommendations for each user.
    \item Assess the observed ranking list with the baseline list and calculate the MAP@10.
\end{enumerate}


\begin{table}[!ht]
    \centering
    \caption{Trained model specifications}
    \label{tab:models}
    \begin{tabular}[!ht]{|l|c|c|c|c|}
        \hline
        \textbf{Model Nr.} & \textbf{Learning rate} & \textbf{Dimensions} & \textbf{Dropout probability} & \textbf{Text normalization} \\
        \hline
        \csvreader[
        late after line=\\\hline
        ]{tables/hyperparameter-models-parameters.csv}
        {
            lr=\lr, 
            dim=\dim, 
            dropoutRHS=\dropoutRHS, 
            normalizeText=\normalizeText
        }
        {
            \thecsvrow & \lr & \dropoutRHS & \dim & \normalizeText
        }%
    \end{tabular}
\end{table}


The experiments use the \emph{Most Interactions} data fusion approach. The reason is that this method was resulting in a better recommendation quality compared to the other two approaches when partitioning the data with the StarSpace model. The experiments were also done once with 500 and 2000 sampled users, respectively, to investigate the impact of the number of sampled users on the overall MAP@10 value.

Before diving into the recommendation quality benchmark of each model, let's see how the different trained models impact the data distribution. In the diagram \ref{plot:edge-distribution} you can see how each model distributes the edges over two partitions. Compared to the data distribution of the Murmur2 hash partitioning method in the diagram \ref{plot:edge-distribution-2-partitions-murmur2} the partitions are unbalanced.


\begin{figure}[h!]
    \centering
    \resizebox{0.8\textwidth}{!}{%
\begin{tikzpicture}
	\begin{axis}[
			xbar,
			xmin=0,xmax=11884898,
			height=19cm, width=13cm,
			enlarge y limits=0.05,
			xlabel={\#Edges},
			ylabel={Model Nr.},
			xtick={0,5942449,11884898},
			scaled x ticks=false,
			xticklabel style={
				/pgf/number format/fixed,
				/pgf/number format/precision=5
			},
			ytick=data,
			nodes near coords, 
			nodes near coords align={horizontal},
			every node near coord/.append style={
				/pgf/number format/fixed,
				/pgf/number format/precision=5
			},
			legend style={at={(0.5,-0.10)},
				anchor=north,legend columns=-1},
		]
		\addplot
		table [y index=0, x index=1] {recommendation/hyperparameter-edge-distribution.dat};
		\addplot 
		table [y index=0, x index=2] {recommendation/hyperparameter-edge-distribution.dat};
		\legend{Partition 0,Partition 1}
	\end{axis}
\end{tikzpicture}
}%
    \caption{Edge distribution of test dataset; sharded with different trained models over two partitions}
    \label{plot:edge-distribution}
\end{figure}


Diagram \ref{plot:hyperparameter-recommendation-quality} shows the computed mean average precision at cut-off ten for different trained models compared to the single partition and the Murmur2 hash function over 500 users. The Highest Interest approach is used to fusion the data together. Model 15 has the highest MAP@10 value for the merged values of two partitions among all other trained models, followed by model 7 with a value of 0.467. The chart \ref{plot:edge-distribution} shows that the data distribution of model 7 is better than model 15. In further evaluations, model 7 is used.

\begin{figure}[h!]
    \centering
    \resizebox{0.8\textwidth}{!}{%
\begin{tikzpicture}
	\begin{axis}[
			xbar,
            xbar=5pt,
			xmin=0,xmax=1,
			height=19cm, width=13cm,
            bar width=0.1cm,
			enlarge y limits=0.03,
			xlabel={MAP@10},
			ylabel={Partitioning Method},
			xtick={0,0.25,0.5,0.75,1},
			scaled x ticks=false,
			xticklabel style={
				/pgf/number format/fixed,
				/pgf/number format/precision=5
			},
			ytick=data,
            symbolic y coords={Single,Murmur2,Model-1,Model-2,Model-3,Model-4,Model-5,Model-6,Model-7,Model-8,Model-9,Model-10,Model-11,Model-12,Model-13,Model-14,Model-15,Model-16},
			nodes near coords, 
			nodes near coords align={horizontal},
			every node near coord/.append style={
				/pgf/number format/fixed,
				/pgf/number format/precision=3,
                font=\footnotesize
			},
			legend style={at={(0.5,-0.09)},
				anchor=north,legend columns=-1},
		]
		\addplot
		table [y index=0, x index=1] {recommendation/hyperparameter-map-at-10.dat};

		\addplot[red, fill=red!30!white, x filter/.expression={x==0 ? nan : x}]
		table [y index=0, x index=2] {recommendation/hyperparameter-map-at-10.dat};

        \addplot[brown, fill=brown!30!white, x filter/.expression={x==0 ? nan : x}]
		table [y index=0, x index=3] {recommendation/hyperparameter-map-at-10.dat};

		\legend{Partition 0,Partition 1, most interactions}
	\end{axis}
\end{tikzpicture}
}%
    \caption{Comparing recommendation quality of different partitioning methods and models for 500 users}
    \label{plot:hyperparameter-recommendation-quality}
\end{figure}


As explained in the section \ref{subsec:experiment-setup} the hyperparameter tests were also done over 2000 users to see if the number of sampled users affects the MAP value. The experiments do not significantly increase or decrease in the overall MAP value in each model and partitioning method. This can be seen in the chart \ref{plot:hyperparameter-recommendation-quality-2000-users}. Therefore all other experiments are done over 500 sampled users.


\begin{figure}[h!]
    \centering
    \begin{tikzpicture}
	\begin{axis}[
			xbar,
            xbar=5pt,
			xmin=0,xmax=1,
			height=19cm, width=13cm,
            bar width=0.1cm,
			enlarge y limits=0.03,
			xlabel={MAP@10},
			ylabel={Partitioning Method},
			xtick={0,0.25,0.5,0.75,1},
			scaled x ticks=false,
			xticklabel style={
				/pgf/number format/fixed,
				/pgf/number format/precision=5
			},
			ytick=data,
            symbolic y coords={Single,Murmur2,Model-1,Model-2,Model-3,Model-4,Model-5,Model-6,Model-7,Model-8,Model-9,Model-10,Model-11,Model-12,Model-13,Model-14,Model-15,Model-16},
			nodes near coords, 
			nodes near coords align={horizontal},
			every node near coord/.append style={
				/pgf/number format/fixed,
				/pgf/number format/precision=3,
                font=\footnotesize
			},
			legend style={at={(0.5,-0.09)},
				anchor=north,legend columns=-1},
		]
		\addplot
		table [y index=0, x index=1] {recommendation/hyperparameter-map-at-10-2000-users.dat};

		\addplot[red, fill=red!30!white, x filter/.expression={x==0 ? nan : x}]
		table [y index=0, x index=2] {recommendation/hyperparameter-map-at-10-2000-users.dat};

        \addplot[brown, fill=brown!30!white, x filter/.expression={x==0 ? nan : x}]
		table [y index=0, x index=3] {recommendation/hyperparameter-map-at-10-2000-users.dat};

		\legend{Partition 0,Partition 1,highest interest}
	\end{axis}
\end{tikzpicture}
    \caption{Comparing recommendation quality of different partitioning methods and models for 2000 users}
    \label{plot:hyperparameter-recommendation-quality-2000-users}
\end{figure}



\subsubsection{Horizontall scalling}
\label{subsubsec:eval-horizontall-scalling}

