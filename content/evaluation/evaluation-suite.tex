\section{Evaluation Suite}
\label{sec:evaluation-suite}
The \emph{Evaluation Suite} evaluates the implementation and the different approaches proposed in this work \footnote{\url{https://github.com/ramin-master-thesis/evaluation}}. The main purpose of the evaluation suite is to measure the recommendation quality of the distributed system. The evaluation suite compares the golden standard with the recommendations generated by the different data fusion approaches (explained in section \ref{sec:data-fusion-approaches}) based on the partitioning method used (see section \ref{sec:partitioning}). Moreover, the evaluation suite can compute the latency of the different system components. The evaluation suite is configurable and automates the evaluation process for different test scenarios. The following section will first go through the overall architecture and explain each building block of the evaluation suite.


\subsection{Architecture and components}
\label{sec:eval-suite-architecture-components}
Figure \ref{fig:evaluation-suite-architecture} indicates an abstract architecture of the evaluation suite and its components. The system takes a configuration file and the left side index of the graph (user to tweets index) as its input. The configuration file provides different parameters to the system to create different evaluation scenarios. The evaluation suite randomly samples users and generates a baseline (golden standard list) recommendation list based on the configuration file passed. Next, the evaluation suite fetches the recommendations from the partitions and computes the recommendation quality along with meta-information (e.g., number of left index vertices, number of edges) gathered from each partition. In the end, the system outputs the benchmark results. Next, each component is described in detail.


\begin{figure}[!h]
    \centering
    \includegraphics[width=0.6\textwidth]{images/evaluation-suite-architecture}
    \caption{High level evaluation suite architecture and its components}
    \label{fig:evaluation-suite-architecture}
\end{figure}


\paragraph{Sampler}
takes the number of samples as an input from the configuration file, samples the elements of the left index side (i.e., users), and then saves them in a file.

\paragraph{Baseline generator}
is responsible for generating the golden standard as the baseline for the evaluation. This component only communicates with the single partition instance and generates the recommendations of the sampled users.

\paragraph{Recommendation fetcher}
communicates with each worker (i.e., partition) and sends the recommendation request of the randomly sampled entities to each machine. It then saves the recommendations in a file.

\paragraph{Recommendation quality calculator}
the quality of the recommendation is measured through this component. This component implements the data fusion approaches described in \ref{sec:data-fusion-approaches} and generates the single ranked list.

Moreover, this component also implements the evaluation metrics \emph{MAP@K} and \emph{RBO} (see section \ref{subsec:evaluation-metrics}) and compares the results with the genrated baseline.

\paragraph{Partition status collector}
collects information about each partition. This information includes the partitioning method, partition-ID, and the number of edges, left and right vertices of the bipartite graph.

