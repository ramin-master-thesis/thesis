\section{Latency}
\label{sec:eval-latency}
The main evaluation focus of this work is on the how the partitioning effects the recommendation qualtiy. But during the evalution of the system, I noticed that the time of recommendation computation in smaller partition sink noteably. Therefore, it was worth measuring it and discusse about the benefetis and the technical reasons of the latency improvement.


The focus of the evaluation here is to measure the time of each instance computing the recommendations using the SALSA algorithm. The assumption here is that by partitioning the data on multiple instances the amount of data on each machine recduces respectively and this will lead to a faster random walk on the bipartite graph. 

\subsection{Hardware specification}
\label{subsec:hardware-spec}
This section describes the machine, where the expriments and evaluation of the proposed thesis took place. It is important to notice the hardware specification specialy when evaluating the latency of the system.

The instance is a VM-Host-Server with 2x AMD EPYC 7282 (Zen-Rome) 16-Core CPU, 120W, 2.80GHz, 64MB L3 Cache, DDR4-3200, Turbo Core max. 3.20GHz. The instance has 256GB (16x 16GB) DDR4-3200 DIMM, REG, ECC, 2R of RAM. The storage is a SSD 1.6TB HHHL NVMe, 24x7, 3 DWPD, Samsung PM1725b. The virtualization is runing with  QEMU-KVM/libvirt version 4.2.1 (Debian 1:4.2-3ubuntu6.17).


\subsection{Experiment strategy}
\label{subsec:latency-experiment-strategy}
The benchmarks are done over one, two, and four partitions. The partitioning method used to distribute the test dataset is \emph{Murmur2 Hash} (see section \ref{subsec:partitioning-murmur2}). Using only this partitioning method is sufficent for evaluating the latency since we can see the relation between the amount of data on each partition and the execution time of the random walk and the data fusion approaches.

To have a percises of the latency the experiments are run ten times and shown on a boxplot. On each run 500 new users are sampled and the recommendations are calculated and then the three data fusion approaches are used to fusion the results.


It is worth mentioning that the code used to simulate GraphJet is not a production code. Thus, these results can be also excepted in a real production-like system. The simulated system is written in Python (version is 3.8) and uses Pandas's (version 1.2.3) DataFrame to store the and retrieve the left, right-side indecies. Moreover, Python's Flask (version 1.1.2) is used to send the request to each partition (i.e., instance).


\subsection{Recommendation Computation Latency}
\label{subsec:recommendation-computation-latency}
This section will go through the experiments done to evaluate the recommendation computation time. Moreover, this section descusess how the amount of data impacts the random walk computation time. The experiments assess the recommendation generation latency of the partitioned data over two, four partitions with the single partition.


The diagrams \ref{plot:edge-distribution-2-partitions-murmur2} and \ref{plot:edge-distribution-4-partitions-murmur2} show the data distribution of the above mentioned partitioning strategy. As you can see in the plots, partitioning the data using Murmur2 hash distributes the edges equaly. 

\begin{figure}[ht!]
    \centering
    \begin{subfigure}{\textwidth}
        \centering
        \begin{tikzpicture}
	\begin{axis}[
			ybar,
			ymin=0,ymax=11884898,
			ylabel={\#Edges},
			xlabel={Hash Function},
			ytick={0,5942449,11884898},
			symbolic x coords={Murmur2},
			scaled y ticks=false,
			yticklabel style={
				/pgf/number format/fixed,
				/pgf/number format/precision=5
			},
			xtick=data,
			bar width=30,
			nodes near coords, 
			node near coords style={
				xshift=0.05cm,
				/utils/exec={\setbox0\hbox{\pgfmathprintnumber\pgfplotspointmeta}
					\pgfmathfloattomacro{\pgfplotspointmeta}{\F}{\M}{\E}
					\pgfmathsetmacro{\myanchor}{
						ifthenelse(\M*pow(10,\E-3)*2-\the\wd0>0,"east","west")
						}
				},
				rotate=90,
				anchor=\myanchor
				},
				every node near coord/.append style={
					/pgf/number format/fixed,
					/pgf/number format/precision=5
				},
				legend style={
					at={(0.5,-0.2)},
					anchor=north,legend columns=-1
				},
				]
				\addplot 
				table [y index=1, x index=0] {latency/murmur2-edge-distribution-2-partitions.dat};
				\addplot 
				table [y index=2, x index=0] {latency/murmur2-edge-distribution-2-partitions.dat};
				\legend{Partition 0,Partition 1}
	\end{axis}
\end{tikzpicture}
        \caption{Edge distribution of over 2 partitions}
        \label{plot:edge-distribution-2-partitions-murmur2}
    \end{subfigure}\qquad

    \begin{subfigure}{\textwidth}
        \centering
        \begin{tikzpicture}
	\begin{axis}[
					ybar,
					ymin=0,ymax=11884898,
					ylabel={\#Edges},
					xlabel={Hash Function},
					ytick={0,5942449,11884898},
					symbolic x coords={Murmur2},
					scaled y ticks=false,
					yticklabel style={
						/pgf/number format/fixed,
						/pgf/number format/precision=0
					},
					xtick=data,
					bar width=30,
					nodes near coords,
					every node near coord/.append style={
						/pgf/number format/fixed,
						/pgf/number format/precision=0,
						xshift=0.2cm,
						yshift=1cm,
						rotate=90,
					},
					legend style={
						at={(0.5,-0.2)},
						anchor=north,legend columns=-1
					},
				]
				\addplot 
				table [y index=1, x index=0] {latency/murmur2-edge-distribution-4-partitions.dat};
				\addplot 
				table [y index=2, x index=0] {latency/murmur2-edge-distribution-4-partitions.dat};
				\addplot
				table [y index=3, x index=0] {latency/murmur2-edge-distribution-4-partitions.dat};
                \addplot 
                table [y index=4, x index=0] {latency/murmur2-edge-distribution-4-partitions.dat};
				\legend{Partition 0, Partition 1, Partition 2, Partition 3}
	\end{axis}
\end{tikzpicture}
        \caption{Edge distribution of over 4 partitions}
        \label{plot:edge-distribution-4-partitions-murmur2}
    \end{subfigure}\qquad
    \caption{Murmur2 hash function edge distribution.}
\end{figure}


After the partitioning process, the latency data is gathered with the respect to the experiment strategy mentioned in section \ref{subsec:latency-experiment-strategy}. The results are demonstrated in the plots \ref{plot:recommendation-latency-two-partitions} and \ref{plot:recommendation-latency-four-partitions}. 

\begin{figure}[ht!]
    \centering
    \begin{subfigure}{\textwidth}
        \centering
        \begin{tikzpicture}
	\begin{axis}[,
			boxplot/draw direction=y,
			x axis line style={opacity=0},
			axis x line*=bottom,
			xtick={1,2,3},
			xticklabels={Single, Partition 0, Partition 1},
			enlarge x limits=0.1,
			xtick distance=2,
			ylabel=Seconds,
			width = 0.70\linewidth,
		]
		\addplot+ [boxplot]
		table [y index=0] {latency/recommendation-latency.dat};

		\addplot+ [boxplot]
		table [y index=1] {latency/recommendation-latency.dat};

		\addplot+ [boxplot]
		table [y index=2] {latency/recommendation-latency.dat};
	\end{axis}
\end{tikzpicture}
        \caption{Single partition recommendation generation latency compared to two partitions}\label{plot:recommendation-latency-two-partitions}
    \end{subfigure}\qquad

    \begin{subfigure}{\textwidth}
        \centering
        \begin{tikzpicture}
	\begin{axis}[,
			boxplot/draw direction=y,
			x axis line style={opacity=0},
			axis x line*=bottom,
			xtick={1,2,3,4},
			xticklabels={Partition 0, Partition 1, Partition 2, Partition 3},
			enlarge x limits=0.1,
			xtick distance=2,
			ylabel=Seconds,
			height = 6cm, width = 12cm
		]

		\addplot+ [boxplot]
		table [y index=1] {latency/recommendation-latency-four-partition.dat};

		\addplot+ [boxplot]
		table [y index=2] {latency/recommendation-latency-four-partition.dat};

		\addplot+ [boxplot]
		table [y index=3] {latency/recommendation-latency-four-partition.dat};

		\addplot+ [boxplot]
		table [y index=4] {latency/recommendation-latency-four-partition.dat};
	\end{axis}
\end{tikzpicture}
        \caption{Recommendation generation latency in four partition}\label{plot:recommendation-latency-four-partitions}
    \end{subfigure}\qquad
    
    \caption{Recommendation generation latency in seconds.}
\end{figure}

The boxplot in \ref{plot:recommendation-latency-two-partitions} indecates the diffrence between the single partition and two partition instance. There is alomost 9.5\% latency improvement when assessing the single partition with partition 0. This improvement gets higher when the amount of partitions increases. The plot in \ref{plot:recommendation-latency-two-partitions} demonstrates this. The latency improves over 40\% with the increase of partitions. This can be explained when we look at the data distribution. Since each partition contains less data, the adjancency lists get smaller and the indecies as well. Therefore, the random walk algorithm can faster retrive data and randomly choose faster between a smaller list for its next node. By increasing the amount of partition the amount of data on each partition recduces respectively and the latency improves.

\subsection{Data Fusion Latency}
\label{subsec:data-fusion-latency}
The multi-partition enviroment uses the data fusion module to generate a single ranked list. Therefore, this module should be also included in latency benchmarks. As regards to the above-mentioned experiment startegy, the results of the experiments show that the data fusion approaches are not a bottle neck and they deliver a "good enough" latency.


The \emph{Most Interactions} method hast the largest latency. That is duet to network calls it needs to do for each user to each partition to gather the degree of the users. If the amount of partitions doubles the latency of this particular approach almost doubles respectively.


The idea of storing the degrees of the user during partitioning time and using a loadbalancer to aske the right partition mentioned in \ref{fig:loadbalancer} could improve the latency. Although, furthure investigations are needed to be done to benchmark the latency of the proposed loadbalancer idea.


\begin{table}[!ht]
    \centering
    \caption{Average latency in seconds of different data fusion approaches}
    \label{tab:data-fusion-latency}
    \begin{tabular}{|c|c|c|}
        \hline
        \textbf{Union Results} & \textbf{Highest Hit} & \textbf{Most Interactions} \\
        \hline
        0.13 & 0.50 & 4.55 \\
        \hline
    \end{tabular}
\end{table}
