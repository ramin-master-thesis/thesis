\section{Evaluation Suit}
\label{sec:evaluation-suit}
The \emph{Evaluation Suit} evaluates the implementation and the different approaches proposed in this work \footnote{\url{https://github.com/ramin-master-thesis/evaluation}}. The main purpose of the evaluation suit is to measure the recommendation quality of the distributed system. The evaluation suit compares the golden standard with the recommendations generated by the different data fusion approaches (explained in section \ref{sec:data-fusion-approaches}) based on the partitioning method used (see section \ref{sec:partitioning}). It also assesses other aspects of the system, like the latency. The evaluation suit is highly configurable and automates the evaluation process for different test scenarios. This section will explain how the evaluation suit is built and how it aids this work to compute the evaluation.


\subsection{Architecture and components}
\label{sec:eval-suit-architecture-components}
Figure \ref{fig:evaluation-suit-architecture} indicates an abstract architecture of the evaluation suite and its components. The system takes a configuration file and the left side index of the graph (user to tweets index) as its input. The configuration file provides different parameters to the system to create different evaluation scenarios. The evaluation suit reads the configuration file and samples the users, generates a baseline (golden standard list) from the sampled users. Next, it fetches the recommendations from the partitions and computes the recommendation quality along with information of each partition. In the end, the system outputs the results. In the follwoing each components functionality is described.


\begin{figure}[!h]
	\centering
	\includegraphics[width=0.6\textwidth]{images/evaluation-suit-architecture}
	\caption{High level evaluation suit architecture and its components}
	\label{fig:evaluation-suit-architecture}
\end{figure}


\paragraph{Sampler}
takes the number of samples as an input from the configuration file, samples the elements of the left index side (i.e., users), and then saves them in a file.

\paragraph{Baseline generator}
is responsible for generating the baseline of the evaluation. This component only communicates with the single partition instance and generates the recommendations of the sampled entities.

\paragraph{Recommendation fetcher}
communicates with each worker (i.e., partition) and sends the recommendation request of the randomly sampled entities to each machine. It then saves the recommendations in a file.

\paragraph{Recommendation quality calculator}
the quality of the recommendation is assessed through this component. This component implements the data fusion approaches described in \ref{sec:data-fusion-approaches} and generates the single ranked list.

Moreover, this component also implements the calculation of the evaluation metrics \emph{MAP@K} and \emph{RBO} (see section \ref{subsec:evaluation-metrics}) and compares the results with the genrated baseline.

\paragraph{Partition status collector}
collects information about each partition. This information includes the partitioning method, partition-ID, and the number of edges, left and right vertices of the bipartite graph.

